\documentclass[12pt]{article}

\usepackage[a4paper,margin=1in]{geometry}
\usepackage{enumitem}
\usepackage{amsmath, amssymb}
\usepackage{hyperref}
\usepackage{graphicx}

\title{Assignment 4: Understanding Wealth Inequality through the Lens of the Aiyagari Model}
\author{Macroeconomics and Inequality (2025)}
\date{Due: Friday 20 June, 9:00}

\begin{document}
\maketitle

Over the past sessions, we have studied the canonical Aiyagari model and its implications for consumption, savings, and wealth accumulation under idiosyncratic risk. In the next class, we will discuss the paper by Hubmer, Krusell, and Smith (2021), which extends this framework to explain long-run and dynamic trends in US wealth inequality.

Please complete the following tasks in preparation for our discussion:

\section*{Part I. Model Comprehension (short answer)}
\begin{enumerate}[label=\textbf{\arabic*.}]
    \item \textbf{Model Innovations} \\
    Identify and explain \textbf{two key ways} in which the authors extend the standard Aiyagari model. Describe how each innovation contributes to matching observed wealth inequality.

    \item \textbf{Portfolio Heterogeneity} \\
    The paper argues that portfolio returns differ systematically across households. Summarise how the authors model this feature and explain why it is important for reproducing both the level and dynamics of wealth inequality.

    \item \textbf{Role of Risk} \\
    Explain how \textbf{increased earnings risk} can lead to \textbf{less wealth inequality} in some parts of the distribution, contrary to what might be expected. Use the model logic to support your answer.
\end{enumerate}

\section*{Part II. Deep Analysis (open-ended writing)}
\begin{enumerate}[start=4, label=\textbf{\arabic*.}]
    \item \textbf{Mechanism Disentanglement} \\
    Choose one of the mechanisms the authors analyse (e.g. changes in tax progressivity, wage inequality, asset return dynamics). Carefully describe how the authors model this mechanism and quantify its effect. Then, \textbf{critically evaluate} the strengths and limitations of this modelling choice. You may contrast it with alternative approaches or interpretations.

    \item \textbf{Data and Calibration} \\
    The model is calibrated using a variety of empirical inputs. Pick \textbf{one specific data input} (e.g. the earnings process, tax rates, or the return schedule on wealth) and describe how it is used in the calibration. Discuss how sensitive the main results might be to alternative empirical estimates for this input.
\end{enumerate}

\section*{Part III. Model Extension (creative design)}
\begin{enumerate}[start=6, label=\textbf{\arabic*.}]
    \item \textbf{Design Your Own Extension} \\
    Propose a specific extension to the model that could capture an additional real-world feature influencing wealth inequality (e.g. intergenerational transfers, health risk, education shocks, etc.).
    \begin{itemize}
        \item Explain how your extension would fit into the existing model structure.
        \item Speculate how it might affect the distributional outcomes.
        \item Briefly outline what data you would use to calibrate your proposed feature.
    \end{itemize}
\end{enumerate}

\subsubsection*{Instructions}

\begin{itemize}
    \item Your answers should be \textbf{your own words} and grounded in the content and logic of the paper.
    \item Use of AI tools is permitted; the grading will reward critical understanding, precise reading, and your own insights.
    \item Be concise and focus on clarity.
    \item Cite specific figures, tables, or passages from the paper when helpful.
\end{itemize}

\end{document}