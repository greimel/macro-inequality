\documentclass[12pt]{article}
\usepackage{amsmath,amssymb}
\usepackage{geometry}
\geometry{margin=1in}
\usepackage{hyperref}

\title{Assignment 5: Wealth Taxation in the Aiyagari Model}
\author{Macroeconomics and Inequality (2025)}
\date{Due: Thursday 26 June, 11:00}

\begin{document}

\maketitle

\section*{Assignment Overview}

This assignment focuses on the Guvenen et al. (2023) paper analyzing the efficiency and redistributional effects of wealth taxation. You will engage with the theoretical model and empirical findings to deepen your understanding of wealth taxation and its implications for inequality and welfare.

\section*{Part I: Understanding the Model and Key Mechanisms}

\begin{enumerate}
    \item Explain the key distinction between a wealth tax and a capital income tax in the presence of heterogeneous returns to wealth. Why does this distinction matter for tax policy?
    \item Describe the "use-it-or-lose-it" mechanism as discussed in the paper. How does it influence the efficiency costs of wealth taxation?
    \item Outline the main features of the model used by Guvenen et al., including the overlapping generations (OLG) structure, entrepreneur heterogeneity, and collateral constraints.
\end{enumerate}

\section*{Part II: Policy Experiments and Welfare Analysis}

\begin{enumerate}
    \item Analyze the revenue-neutral reform that replaces the capital income tax with a wealth tax. What are the key mechanisms through which this reform affects economic outcomes and welfare?
    \item Summarize the findings regarding the optimal wealth tax (OWT) and optimal capital income tax (OKIT). Why is the OWT positive while the OKIT is negative? Discuss the economic intuition behind these results.
\end{enumerate}

\section*{Part III: Extensions and Further Research}

\begin{enumerate}
    \item Propose a potential extension to the model that could capture additional realistic features of wealth taxation, such as progressive tax brackets or the distinction between book and market values of wealth. How might this extension affect the paper’s conclusions?
\end{enumerate}

\subsubsection*{Instructions}

\begin{itemize}
    \item Your answers should be \textbf{your own words} and grounded in the content and logic of the paper.
    \item Use of AI tools is permitted; the grading will reward critical understanding, precise reading, and your own insights.
    \item Be concise and focus on clarity.
    \item Cite specific figures, tables, or passages from the paper when helpful.
\end{itemize}

\end{document}
