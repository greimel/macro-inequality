\documentclass[a4paper,12pt]{article}

%\usepackage{titling}
%\setlength{\droptitle}{-4cm}
%\predate{}
%\postdate{}
%\preauthor{}
%\postauthor{}

\usepackage{xcolor}

\newcommand{\red}{\textcolor{red}}
\newcommand{\orange}{\textcolor{orange}}
\newcommand{\gray}{\textcolor{gray}}
\newcommand{\green}{\textcolor{green}}
\newcommand{\blue}{\textcolor{blue}}

\newcommand{\new}{\red{\emph{NEW!}} }
\newcommand{\maybe}{\orange{\emph{NEW?}} }

\usepackage{amsmath,amssymb,amsthm,mathtools}
\usepackage{bm}
\usepackage{booktabs,hyperref}
\usepackage[longnamesfirst]{natbib}
\bibliographystyle{ecta}

\begin{document}
\noindent {\LARGE Macroeconomics and Inequality}

\medskip \noindent {\large MSc Economics, University of Vienna}

\subsection*{Lecturer}

Fabian Greimel, \url{https://www.greimel.eu}

%\noindent
%Stefanie J. Huber, \url{https://sites.google.com/site/stefaniehuber/}

\subsection*{Description}

The term ``distributional macroeconomics'' was introduced by Benjamin Moll as a replacement for ``macroeconomics with heterogeneous agents'' to promote the view that the \emph{macroeconomy is a distribution} of state variables (e.g.\ income and wealth).\footnote{See \url{ https://benjaminmoll.com/wp-content/uploads/2019/07/DM_long.pdf}.}

This course will explore some implications of income and wealth heterogeneity for macroeconomic dynamics and macroeconomic policy. First, we want to understand how secular trends (falling interest rates and rising debt) are connected to rising inequality. And second, we want to understand how heterogeneity matters for the aggregate response to macroeconomic shocks.

The focus of the course is on topics, rather than computational methods. (These are covered in second year electives.) We will use simplified models to replicate results from papers and compare the results to empirical evidence. For most topics, we will provide ready-to-run code in the form of interactive \href{https://github.com/fonsp/Pluto.jl}{\texttt{Pluto.jl}} notebooks. 

%The course is based on recent research papers. Most papers build on the Bewley-Huggett-Aiyagari model, some build on the two-agent (Saver-Spender) model. 

\subsection*{Topics}

\begin{enumerate}
\item $J$-period lifecycle model
\item Overlapping generations
\item Infinite horizon
\item Income risk, borrowing constraints
\item 
\item 
\item Modelling the wealth distribution
\begin{itemize}
  \item Wealth inequality
  \item Wealth taxes
\end{itemize} 
\item Welfare
\item Housing
\begin{itemize}
  \item Housing 
\end{itemize}
\end{enumerate}

\subsection*{Evaluation}

\begin{enumerate}
\item Assignments
\item Midterm exam
\item Final exam
\end{enumerate}

\subsection*{Structure and Content}

\begin{enumerate}
 \item Household Balance Sheets and the Great Recession
  %  \begin{enumerate}
  %  \item \cite{mian2013household}
  %  \item \cite{berger2015consumption}
  %  \end{enumerate}
 \item Housing Wealth Effects: How Consumption Reacts to a House Price Bust
  %  \begin{enumerate}
  %  \item \cite{berger2018house}
  %  \item \cite{guren2021housing}
  %  \end{enumerate}
%   \item Saving Glut of the Rich  \citep{kumhof2015inequality,mian2021indebted-demand}
   \item Social comparisons and trickle-down consumption
     \citep{bertrand2016trickle, bellet2024mcmansion, drechsel2025falling-behind}
   \item Microeconomics Heterogeneity and Macroeconomic Shocks \citep{kaplan2018microeconomic}
   \item Inequality and growth \citep{moll2022uneven}
   \item \citet{hubmer2021sources} decompose wealth inequality into various sources. Earnings, tax progressivity, and return heterogeneity (Table 2)

   Exercise: How does wealth inequality depend on the tax system. (Progressive taxes lower wealth inequality compared to a flat tax.)

   \item \citet{guvenen2023use}
   \item \citet{fagereng2022asset}
   \item \citet{guvenen2022global} insights from the GRID project
\end{enumerate}


\section{Lecture by lecture}

\subsection{Part 1: Preliminaries}

\subsection{Part 2: Wealth inequality}

\begin{enumerate}
  \item {}[Feature] Income risk and credit constraints (I)
  \item {}[Feature] Income risk and credit constraints (II)
  \item \new Taking $J \to \infty$
  \item Household as a Markov Chain
  \item Modelling the Wealth Distribution
  \item {}[Paper] Wealth inequality \citep{hubmer2021sources}
  \item {}[Paper] \new Housing markets and Wealth inequality (Kindermann)
  \item \new Measuring Social Welfare (for policy evaluation)
  \item \maybe (Housing markets and wealth inequality II (redistribution from renters to landlords?))
  \item {}[Feature] \new Entrepreneurs
  \item {}[Paper] Wealth taxes (I) \citep{guvenen2023use}
  \item {}[Paper] Wealth taxes (II) \citep{boar2023income-or-wealth}
\end{enumerate}

\subsection{Part 3: Bonus topics}

\begin{enumerate}
  \item {} [Paper] \maybe Microeconomics Heterogeneity and Macroeconomic Shocks \citep{kaplan2018microeconomic}
  \item {} [Paper] \maybe CO$_2$ tax
\end{enumerate}



% \section{Models \& Code}

% \begin{enumerate}
% \item Consumer debt and default [Q]
% \item Simple Model of Housing [A]
% \item Housing and Renting [Q] (cf slides)
% \item Discrete-time vs continuous time (infinitesimal operators bla bla)
% \end{enumerate}


\bibliography{../inequality.bib}
\end{document}
